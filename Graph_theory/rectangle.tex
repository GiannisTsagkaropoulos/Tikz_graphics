\documentclass[border=10pt]{standalone}
\usepackage{tikz}
\usetikzlibrary{shapes.geometric, calc}

\begin{document}

  \begin{tikzpicture}[line cap=round]
        \def\r{80pt} %radius of circumscribed circle
        \def\rn{90pt} %radius of "invincible" circle with edges numbered 
        \pgfmathtruncatemacro\a{12} % number of polygon vertices
                

\foreach \i in {1, 2, ..., \a} {
    \coordinate (p\i) at ({-\r*cos(\i*(360/\a) + (90-360/\a)},{\r*sin(\i*(360/\a) + (90-360/\a))});
    \draw[fill=black] (p\i) circle (1pt); 
              
\node at (
   {-\rn*cos(\i*(360/\a) + (90-360/\a)},
   {+\rn*sin(\i*(360/\a) + (90-360/\a))}
){\i};
}

%Polygon Edges
\pgfmathtruncatemacro\b{\a-1} %used to iterate till \a-1
\foreach \i in {1, 2, ..., \b} {
    \pgfmathsetmacro{\j}{\i+1}
    \draw[thick,black] (p\i) -- (p\j);
}   
\draw[thick,black] (p\a) -- (p1);
            
  \end{tikzpicture}
\end{document}